\title{Comparison of Sequential and Parallel Fast Fourier Transform}
\author{
	Rami Jurdi \\
	Zachary Noble \\
	Gregory Freitas \\
	Jayden Bendezu
}
\date{\today}

\documentclass[journal]{IEEEtran}
\usepackage[utf8]{inputenc}
\usepackage{amsmath}

\begin{document}
\maketitle

\begin{abstract}
This research is in the Digtial Signal Processing field. Where our aims are to improve the performance of realtime signal processing using parallel processing techniques coupled with 1 dimensional Fast Fourier Transforms.  It has been done before by other researchers implementing multi-dimensional Fast Fourier Transforms in a multithreadeed context. The purpose of our research is to gain a better understanding of parallel processing techniques and digital signal processing. Thus the main goal is to observe the outcome of implementing the multithreaded Fast Fourier transform algorithm and learn from the state-of-the-art research.
\end{abstract}

\section{Introduction}
	\par {The Fast Fourier transform (FFT) is an algorithm that uses Discrete Fourier transforms
	 (DFT) on a time sequence to convert a signal, usually based on time, to a signal based in the
	 frequency domain. This allows us to analyze a sequence of time-values by decomposing it into
	 bins of different frequencies. The Fast Fourier transform is used in many applications ranging
	 from digital signal processing, sampling, pitch correction software, and wave analysis.}

	\par {The direct computation of the DFT results in $n^2$ multiplications and $n(n-1)$ additions,
	making the computation greatly expensive for sufficiently large $n$.  Over the course of many years
	of research done by researchers in the field, more efficient algorithms were developed for computing DFTs,
	such as the Cooley-Tukey FFT algorithm. This algorithm reduces the time complexity of the computation of
	Fourier transforms from $O(n^2)$ to $O(nlogn)$~\cite{Xiang}}.

	\par {In this paper, we present our process of implementing the parallel FFT algorithm, specifically
	the Cooley-Tukey algorithm, and conduct performance comparisons between the sequential and
	parallel versions of the algorithm in C++.}

\section{Target Problem}
	\par {The main problem at hand is to create an application that is able to process signals
	provided to the program as audio files such as .WAV, .MP3, and .MP4. Then displaying the audio
	files as a spectrograph of its signals. This problem can be broken into several steps of
	what we need to acheive.}

\begin{itemize}
	\item A GUI to interact with.
	\item Accepting audio files via local upload or recording.
	\item Processing the audio files.
	\item Processing the signals for audio files.
	\item Displaying the signals of the audio files on a spectrograph.
\end{itemize}

\subsection{Approach}
	\par {Tackling some of the above sub-problems. In order to create a GUI to interact with,
	we will write the program using the QT C++ GUI framework to simplify taking audio files
	from disk and decoding the raw data. Through the use of provided libraries of QT the
	basic functionality of loading files and playing/decoding audio is handled and we use
	them on a higher level. This will allow us to focus more on the problem of processing
	signals for the audio files. Then by handling extracting raw data from the audio files
	using the QAudioDecoder class provided by QT, we intend to use one-dimensional Fast
	Fourier Transforms to process the signals. (How do we plan to parallelize the FFT is
	what would be discussed here briefly since will be explained thoroughly within
	the algorithms subsection).}
\subsubsection{Project Architecture}
    \par{The core of the project resides in the FTController, which spawns threads using a queue to hold each one.}
    \par{For the distributed DFT, the number of samples is calculated and if the result is 0, it exits. The raw data is taken as a char array and represented as a short array as well. Following is the calculation for the range of data that the thread will process. Each sample is evaluated where the raw data stored in the short array is accessed and multiplied by the real and imaginary values that are represented as cosine and sine respectfully. This product is added to the current sum, a complex double.}
    \par{The magnitude of the current sum is evaluated and stored as the maximum sum if it is the largest y-value seen so far. The point of the graph is stored as an (x,y) coordinate in the vector output.}
    \par{The output vector is iterated only on the lower half, where the frequencies that are within the scope of our desired frequency are plotted in qt.}
    \par{For the distributed FFT, the Cooley-Turkey FFT algorithm is utilized. First, the length of the real and imaginary vectors are checked to be of the same length. Once that is verified the number of levels for the FFT are computed by bit shifting the length. If the length is not a power of 2, then both the real and imaginary vectors must be resized to the next highest power of 2. }
    \par{The sine and cosine calculations are done next, followed by bit-reversing the addressing permutation. Next the Cookey-Tukey decimation-in-time radix-2 FFT algorithm is done. Finally the output vector is filled by calculating the magnitude of the complex element and updating the maxSum atomic member variable with the greatest magnitude.  The corresponding frequency bin from the current index is delivered from the FFTUtils and if the frequency is a duplicate from a previous index it is skipped. The frequency and amplitude are added to the output vector as a pair.}
    \par{The FTController is able to terminate running threads by requesting an interruption and making the thread wait. If the thread notices that an interruption was requested, it will clear its data.  This check occurs when the thread is iterating through each sample.}

\subsection{Plan Outline}

\begin{enumerate}
	\item Setup programming environment using Qt C++.
	\item Add support using Qt multimedia API's to load and decode audio files from disk.
	\item Stream audio data to an output device such as speakers or headphones.
	\item Display a waveform of the audio.
	\item Implement single-threaded Fast Fourier Transforms.
	\item Implement multi-threaded Fast Fourier Transforms.
	\item Display the frequency vs. amplitude data calculated from the Fourier transform.
	\item Conducting experimental tests comparing multi-threaded implementation vs sequential implementation.
Observing any performance boosts if any.
\end{enumerate}

\subsection{Algorithm}

\subsubsection{Discrete Fourier Transform}
% can go into just what the disrete fourier transform is and its runtime and space complexity.
	\par While we established that we will be using the Fourier Transforms. As the base implementation of fourier transforms resolves to a $O(n^2)$ runtime according to~\cite{Xie}.
\subsubsection{Fast Fourier Transform}
% can go into just what the fast fourier transform is and its runtime and space complexity.
	\par It is essential that we use an implementation that yields a better runtime as in the case of our program, operating on larger audio files will begin to take much longer being inefficient. Thus we will be working with the Fast Fourier transfom algorithm to reduce our upper asymptotic bound.  (Going to find a reference for this line to talk more about the $O(nlog{}n)$ implementation. But i need to read more).
\subsection{Paralellized Fast Fourier Transform}
% Can specify more on how the threads will interact with the fourier transform, why the number of threads we chose and such.

\subsection{Experimental Results}
Creating testing data sets \dots

\begin{tabular} { |c|c|c|c| }
	\hline
	\multicolumn{4} {|c|} {Data Set 1} \\
	\hline
	Algorithm & File Type & File Size & Computation Time \\
	\hline
	Sequential FFT & foo.WAV & 100 kb & N \\
	Parallel FFT & foo.WAV & 100 kb & N \\
	\hline
\end{tabular}


\begin{tabular} { |c|c|c|c| }
	\hline
	\multicolumn{4} {|c|} {Data Set 2} \\
	\hline
	Algorithm & File Type & File Size & Computation Time \\
	\hline
	Sequential FFT & foo.WAV & 100 kb & N \\
	Parallel FFT & foo.WAV & 100 kb & N \\
	\hline
\end{tabular}

\begin{tabular} { |c|c|c|c| }
	\hline
	\multicolumn{4} {|c|} {Data Set 3} \\
	\hline
	Algorithm & File Type & File Size & Computation Time \\
	\hline
	Sequential FFT & foo.WAV & 100 kb & N \\
	Parallel FFT & foo.WAV & 100 kb & N \\
	\hline
\end{tabular}

\section{State-of-the-Art Research}

	\par {The papers that we will use to implement the parallel 1D Fast Fourier
	Transform algorithm are the \textit{Proceedings of the Thirty-Seventh
	Southeastern Symposium on System Theory}~\cite{Al},
	the ..............}

\section{Related Work}
\par{There are various applications of the Discrete Fourier Transform in different fields of study. We found topics ranging from digital image processing to modeling the earth's surface.}
\par{When conducting remote sensing from satellites for modeling the earth's surface, one of the most important problems is modeling surface—fragments of dynamic surfaces such as the waves of the ocean and land forms that slowly shift as well. The use of a Discrete Fourier Transform is recommended by some researchers, where the timing of implementing the transform is crucial for improving the quality and accuracy. The use of parallel computations helps to decrease the implementation time. By decreasing the length of time, more of the model surface-fragment can be generated, which would improve the accuracy and quality of the overall model.}
\cite{book:GIS}
\par{Another related research paper that was found had a focus on the possible parallelization techniques for multidimensional hypercomplex discrete Fourier transform (HDFT). The HDFT has been mainly utilized in image and multidimensional signal processing and can be given by the following algebraic expression:
\[ \sum_{n_1,...,n_d=0}^{N-1} f(n_1,...,n_d)W^{<m,n>} = \prod_{k=1}^{d} w_k^{m_kn_k}, w_k^{N} = 1\]
One notable aspect of the transform is that the N-th roots $w_{k}$ from unity are found in different sub-algebras that are isomorphic to a complex algebra of some $2^{d}$ algebra $B_{d}$. Computing a multidimensional transform takes great effort when the dimensionality increases. In generating the hypercomplex spectrum, an analog of the inner parallelism of the Cooley-Tukey scheme was used.}
\cite{Chicheva}
\par{Another related publishing had a focus on creating a more accurate calculation of the Discrete Fourier Transform. In some cases the Discrete Fourier Transform is insufficient in approximating the continuous Fourier transform. For example, a function such as $h (t) = e^{-50t}, t\in [0,1]$, the error on DFT \{h\} around f = 64 decreases to approximately $N^{-1/3}$. Which means that N must increase by a factor of 1000 in order to decrease the error by a factor of 10. The paper presented a method to provide accurate approximations of the continuous Fourier transform with a similar time complexity to the Fast Fourier Transform. The assumption of signal periodicity is no longer rigid and allows to compute numerical Fourier transforms in a broader domain of frequency than the usual half-period of the DFT. This behavior is highly recommended in image processing since it allows to obtain the Fourier transform of an image without the usual interference of the periodicity of the classical DFT.}
\cite{Beaudoin}

\section{Our Contributions}

% test \parencite{1dFft}
\medskip
\bibliographystyle{ieeetr}
\bibliography{report.bib}

\end{document}
