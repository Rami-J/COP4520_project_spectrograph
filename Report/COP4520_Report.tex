\title{Spectrograph COP4520}
\author{
	Rami Jurdi \\
	Zachary Noble \\
	Gregory Freitas \\
	Jayden Bendezu
}
\date{\today}


\documentclass[journal]{IEEEtran}
\usepackage[backend=biber, style=authoryear-icomp]{biblatex}
\usepackage[utf8]{inputenc}
\addbibresource{report.bib}


\begin{document}
\maketitle

\begin{abstract}
This research is in the Digtial Signal Processing field. Where our aims are to improve the performance of realtime
 signal processing using parallel processing techniques coupled with 1 dimensional Fast Fourier Transforms.
 It has been done before by other researchers implementing multi-dimensional Fast Fourier Transforms in a
 multithreadeed context. The purpose of our research is to gain a better understanding of parallel processing
 techniques and digital signal processing. Thus the main goal is to observe the outcome of implementing the
 multithreaded Fast Fourier transform algorithm and learn from the state-of-the-art research.
\end{abstract}

\section{Target Problem}
The main problem at hand is to create an application that is able to process signals provided to the program as audio 
files such as .WAV, .MP3, and .MP4. Then displaying the audio files as a spectrograph of its signals. This problem 
can be broken into several steps of what we need to acheive.
\begin{itemize}
	\item A GUI to interact with.
	\item Accepting audio files via local upload or recording.
	\item Processing the audio files.
	\item Processing the signals for audio files.
	\item Displaying the signals of the audio files on a spectrograph.
\end{itemize}

\subsection{Approach}
Tackling some of the above sub-problems. In order to create a GUI to interact with, we will write the program using 
the QT C++ GUI framework to simplify taking audio files from disk and decoding the raw data. Through the use of provided libraries of QT 
the basic functionality of loading files and playing/decoding audio is handled and we use them on a higher level. This will 
allow us to focus more on the problem of processing signals for the audio files. Then by handling extracting 
raw data from the audio files using the QAudioDecoder class provided by QT, we intend to use one-dimensional Fast Fourier Transforms to 
process the signals. (How do we plan to parllelize the FFT is what would be discussed here briefly since will be 
explained througouly within the algorithms subsection).

\subsection{Plan Outline} (Temporary to keep this in)

\begin{enumerate}
	\item Setup programming enviornment using Qt C++.
	\item Add support using Qt multimedia API's to load and decode audio files from disk.
	\item Stream audio data to an output device such as speakers or headphones.
	\item Display a waveform of the audio.
	\item Implement single-threaded Fast Fourier Transforms.
	\item Implement multi-threaded Fast Fourier Transforms.
	\item Display the frequency vs. amplitude data calculated from the Fourier transform.
	\item Conducting experimental tests comparing multi-threaded implementation vs sequential implementation. 
Observing any performance boosts if any.
\end{enumerate}

\subsection{Algorithm}
Will speak about what type of algorithm we intend to use to parallelize it. Discuss it in depth and even evaluate 
expected Big O runtime of our implementation. 

\subsection{Experimental Results}
Here I imagine we can create some test data sets using tables.

\begin{tabular} { |c|c|c|c| }
\hline
\multicolumn{4} {|c|} {Data Set 1} \\
\hline
Algorithm & File Type & File Size & Computation Time \\
\hline
Sequential FFS & foo.WAV & 100 kb & N \\
Parallel FFS & foo.WAV & 100 kb & N \\
\hline
\end{tabular}


\begin{tabular} { |c|c|c|c| }
\hline
\multicolumn{4} {|c|} {Data Set 2} \\
\hline
Algorithm & File Type & File Size & Computation Time \\
\hline
Sequential FFS & foo.WAV & 100 kb & N \\
Parallel FFS & foo.WAV & 100 kb & N \\
\hline
\end{tabular}

\begin{tabular} { |c|c|c|c| }
\hline
\multicolumn{4} {|c|} {Data Set 3} \\
\hline
Algorithm & File Type & File Size & Computation Time \\
\hline
Sequential FFS & foo.WAV & 100 kb & N \\
Parallel FFS & foo.WAV & 100 kb & N \\
\hline
\end{tabular}

\section{State-of-the-Art Research}

\section{Related Work}

\section{Our Contributions}

% This was just for testing biber, as I have never used before...

% As \textcite{1dFft} says, there some weird stuff.

% test \parencite{1dFft}

\printbibliography

\end{document}



