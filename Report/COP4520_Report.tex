\title{Comparison of Sequential and Parallel Fast Fourier Transform}
\author{
	Rami Jurdi \\
	Zachary Noble \\
	Gregory Freitas \\
	Jayden Bendezu
}
\date{\today}

\documentclass[journal]{IEEEtran}
\usepackage[utf8]{inputenc}

\begin{document}
\maketitle

\begin{abstract}
This research is in the Digtial Signal Processing field. Where our aims are to improve the performance of realtime signal processing using parallel processing techniques coupled with 1 dimensional Fast Fourier Transforms.  It has been done before by other researchers implementing multi-dimensional Fast Fourier Transforms in a multithreadeed context. The purpose of our research is to gain a better understanding of parallel processing techniques and digital signal processing. Thus the main goal is to observe the outcome of implementing the multithreaded Fast Fourier transform algorithm and learn from the state-of-the-art research.
\end{abstract}

\section{Introduction}
	\par {The Fast Fourier transform (FFT) is an algorithm that uses Discrete Fourier transforms
	 (DFT) on a time sequence to convert a signal, usually based on time, to a signal based in the
	 frequency domain. This allows us to analyze a sequence of time-values by decomposing it into 
	 bins of different frequencies. The Fast Fourier transform is used in many applications ranging 
	 from digital signal processing, sampling, pitch correction software, and wave analysis.}

	\par {The direct computation of the DFT results in $n^2$ multiplications and $n(n-1)$ additions, 
	making the computation greatly expensive for sufficiently large $n$.  Over the course of many years
	of research done by researchers in the field, more efficient algorithms were developed for computing DFTs,
	such as the Cooley-Tukey FFT algorithm. This algorithm reduces the time complexity of the computation of
	Fourier transforms from $O(n^2)$ to $O(nlogn)$~\cite{Xiang}}.

	\par {In this paper, we present our process of implementing the parallel FFT algorithm, specifically
	the Cooley-Tukey algorithm, and conduct performance comparisons between the sequential and 
	parallel versions of the algorithm in C++.}

\section{Target Problem}
	\par {The main problem at hand is to create an application that is able to process signals 
	provided to the program as audio files such as .WAV, .MP3, and .MP4. Then displaying the audio 
	files as a spectrograph of its signals. This problem can be broken into several steps of 
	what we need to acheive.}

\begin{itemize}
	\item A GUI to interact with.
	\item Accepting audio files via local upload or recording.
	\item Processing the audio files.
	\item Processing the signals for audio files.
	\item Displaying the signals of the audio files on a spectrograph.
\end{itemize}

\subsection{Approach}
	\par {Tackling some of the above sub-problems. In order to create a GUI to interact with, 
	we will write the program using the QT C++ GUI framework to simplify taking audio files 
	from disk and decoding the raw data. Through the use of provided libraries of QT the 
	basic functionality of loading files and playing/decoding audio is handled and we use 
	them on a higher level. This will allow us to focus more on the problem of processing 
	signals for the audio files. Then by handling extracting raw data from the audio files 
	using the QAudioDecoder class provided by QT, we intend to use one-dimensional Fast 
	Fourier Transforms to process the signals. (How do we plan to parllelize the FFT is 
	what would be discussed here briefly since will be explained througouly within 
	the algorithms subsection).}
\subsubsection{Project Architecture}


\subsection{Plan Outline}

\begin{enumerate}
	\item Setup programming environment using Qt C++.
	\item Add support using Qt multimedia API's to load and decode audio files from disk.
	\item Stream audio data to an output device such as speakers or headphones.
	\item Display a waveform of the audio.
	\item Implement single-threaded Fast Fourier Transforms.
	\item Implement multi-threaded Fast Fourier Transforms.
	\item Display the frequency vs. amplitude data calculated from the Fourier transform.
	\item Conducting experimental tests comparing multi-threaded implementation vs sequential implementation. 
Observing any performance boosts if any.
\end{enumerate}

\subsection{Algorithm}
	 
\subsubsection{Discrete Fourier Transform}
% can go into just what the disrete fourier transform is and its runtime and space complexity.
	\par While we established that we will be using the Fourier Transforms. As the base implementation of fourier transforms resolves to a $O(n^2)$ runtime according to~\cite{Xie}.
\subsubsection{Fast Fourier Transform}
% can go into just what the fast fourier transform is and its runtime and space complexity.
	\par It is essential that we use an implementation that yields a better runtime as in the case of our program, operating on larger audio files will begin to take much longer being inefficient. Thus we will be working with the Fast Fourier transfom algorithm to reduce our upper asymptotic bound.  (Going to find a reference for this line to talk more about the $O(nlog{}n)$ implementation. But i need to read more).
\subsection{Paralellized Fast Fourier Transform}
% Can specify more on how the threads will interact with the fourier transform, why the number of threads we chose and such.

\subsection{Experimental Results}
Creating testing data sets \dots

\begin{tabular} { |c|c|c|c| }
	\hline
	\multicolumn{4} {|c|} {Data Set 1} \\
	\hline
	Algorithm & File Type & File Size & Computation Time \\
	\hline
	Sequential FFT & foo.WAV & 100 kb & N \\
	Parallel FFT & foo.WAV & 100 kb & N \\
	\hline
\end{tabular}


\begin{tabular} { |c|c|c|c| }
	\hline
	\multicolumn{4} {|c|} {Data Set 2} \\
	\hline
	Algorithm & File Type & File Size & Computation Time \\
	\hline
	Sequential FFT & foo.WAV & 100 kb & N \\
	Parallel FFT & foo.WAV & 100 kb & N \\
	\hline
\end{tabular}

\begin{tabular} { |c|c|c|c| }
	\hline
	\multicolumn{4} {|c|} {Data Set 3} \\
	\hline
	Algorithm & File Type & File Size & Computation Time \\
	\hline
	Sequential FFT & foo.WAV & 100 kb & N \\
	Parallel FFT & foo.WAV & 100 kb & N \\
	\hline
\end{tabular}

\section{State-of-the-Art Research}

	\par {The papers that we will use to implement the parallel 1D Fast Fourier 
	Transform algorithm are the \textit{Proceedings of the Thirty-Seventh 
	Southeastern Symposium on System Theory}~\cite{Al}, 
	the ..............}

\section{Related Work}
\par{Here we will talk about how our work relates to currently worked on research.}

\section{Our Contributions}

% This was just for testing biber, as I have never used before...

% As \textcite{1dFft} says, there some weird stuff.

% test \parencite{1dFft}
\medskip
\bibliographystyle{ieeetr}
\bibliography{report.bib}

\end{document}



